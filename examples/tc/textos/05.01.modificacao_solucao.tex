\section{Modificação da solução proposta}
\label{modificacao-solucao}

A proposta de solução original deste trabalho estabeleceu que seriam realizadas modificações no código-fonte do \gls{sgbd} MySQL a fim de permitir a simulação de índices sem a necessidade de criar sua estrutura completa no banco de dados. Esta decisão em fase de projeto, leia-se TC I, foi dada com base em soluções semelhantes já desenvolvidas para outros \glspl{sgbd} do mercado (SQL Server, Oracle, DB2, entre outros). A partir desta simulação, pretendia-se obter as performances de processamento com os índices sugeridos pelos algoritmos implementados, utilizando os resultados destes testes para gerar a recomendação final da ferramenta.

Depois de uma longa dedicação para conhecer o código-fonte do MySQL, principalmente no que tange à seleção de índices para execução de consultas e seus cálculos de performance, não foi obtido êxito nesta tarefa. Neste período, observou-se que os trabalhos relacionados são frutos de longas pesquisas, fato que não havia sido estimado. Desta forma, resolveu-se modificar o projeto de solução deste trabalho, como descrito na sequência deste capítulo.

As seções a seguir detalham as etapas de desenvolvimento e execução da ferramenta. Na seção \ref{arquitetura-interna} é apresentada de forma sucinta a arquitetura interna do ambiente desenvolvido; na seção \ref{entrada-de-dados} é descrita a forma em que os dados de entrada devem ser apresentados à ferramenta para análise; em \ref{geracao-indices-candidatos} é apresentado o algoritmo que foi implementado para gerar um conjunto de possíveis índices para as consultas examinadas; a seção \ref{verificacao-custo-indices-candidatos} descreve o método que foi utilizado para estimar o custo das consultas com os índices candidatos gerados na etapa anterior; a seção \ref{geracao-solucao-final} apresenta o procedimento realizado para se obter a solução final com a escolha dos índices que apresentam o menor custo total para o banco de dados.

